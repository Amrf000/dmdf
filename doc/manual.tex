\documentclass[a4paper,12pt,titlepage]{report}
\usepackage[utf8]{inputenc}
\usepackage{graphicx}
\usepackage{lscape}
\usepackage{listings}
\usepackage{float}
\usepackage{color}
\usepackage[paper=a4paper,left=25mm,right=25mm,top=25mm,bottom=25mm]{geometry}
\usepackage{listingsutf8} % UTF-8-Unterstützung, kein \lstset mit iterate wird mehr benötigt
% clear type-Schriften, Output Encoding UTF-8, damit kopieren von Umlauten funktioniert.
\usepackage[T1]{fontenc}
% deutsche Anführungszeichen direkt benutzen
% http://de.wikibooks.org/wiki/LaTeX-W%C3%B6rterbuch:_Anf%C3%BChrungszeichen
\usepackage[ngerman]{babel}
\usepackage[babel,german=quotes]{csquotes}
% mit \enquote{ einheitlich!
% für Zitate
%\usepackage[round]{natbib}

\PassOptionsToPackage{hyphens}{url} % Better line breaks in URLs http://tex.stackexchange.com/questions/3033/forcing-linebreaks-in-url
\usepackage[pdftex,
 pdfcreator=pdflatex]{hyperref}
\hypersetup{
	pdfauthor={Tamino Dauth},
	pdftitle={The Power of Fire Developer Manual},
	pdfsubject={The Power of Fire Developer Manual},
	pdfkeywords={warcraft3,thepoweroffire},
	pdfproducer={Tamino Dauth},
	breaklinks=true
}

%\usepackage{calculator} % calculate mathematical expressions
\usepackage{alltt} % for <pre> like quotes

\usepackage[]{fancyhdr}


% für Literaturverzeichnis
% https://tex.stackexchange.com/questions/51434/biblatex-citation-order
% Sortierreihenfolge und andere Daten stammen von folgender Seite:
% http://www.studis-online.de/Studieren/Wissenschaftliche_Texte/literaturverzeichnis.php
\usepackage[
backend=biber,
natbib=true,
style=numeric,
sorting=none,
maxbibnames=99 % print all authors http://tex.stackexchange.com/questions/1554/biblatex-displaying-all-authors-of-multi-author-works-in-the-bibliography
]{biblatex}

% eigner Style
% Beispiel: http://ctan.space-pro.be/tex-archive/macros/latex/contrib/biblatex-contrib/uni-wtal-ger/uni-wtal-ger.bbx
% https://tex.stackexchange.com/questions/12806/guidelines-for-customizing-biblatex-styles

% We use the database biblatex-examples.bib.
\addbibresource{Bibliographie.bib}

%\AtEveryBibitem{
%	\ifentrytype{manual}{Handbuch: }
%	{
%		\ifentrytype{article}{Artikel: }
%		{
%			\ifentrytype{online}{Online: }
%			{
%				\ifentrytype{book}{Buch: }
%				{Paper: }
%			}
%		}
%	}
%}

%\definecolor{mygray}{rgb}{0.76,0.76,0.76}
\definecolor{mygray}{rgb}{0.5,0.5,0.5}
\definecolor{mymauve}{rgb}{0.58,0,0.82}
\definecolor{mygreen}{rgb}{0,0.4,0}
% Go highlighting:
% http://tex.stackexchange.com/questions/63496/how-to-add-go-to-the-programming-languages-list-in-lyx
% Unter Gentoo:
% /usr/share/texmf-dist/tex/latex/listings/


% "define" JASS
%Keyword list taken from the scaladoc definition.
\lstdefinelanguage{JASS}{
  morekeywords={%
          function,takes,nothing},
  otherkeywords={=>,<-,<\%,<:,>:,\#,@},
  sensitive=true,
  morecomment=[l]{//},
  morecomment=[n]{/*}{*/},
  morestring=[b]",
  morestring=[b]',
  morestring=[b]"""
}[keywords,comments,strings]



\lstset{ %
language=[]JASS, % [11]                % choose the language of the code
morekeywords={override},
basicstyle=\footnotesize,       % the size of the fonts that are used for the code
numbers=left,                   % where to put the line-numbers
numberstyle=\footnotesize\color{mygray},      % the size of the fonts that are used for the line-numbers
stepnumber=1,                   % the step between two line-numbers. If it is 1 each line will be numbered
numbersep=5pt,                  % how far the line-numbers are from the code
backgroundcolor=\color{white},  % choose the background color. You must add \usepackage{color}
showspaces=false,               % show spaces adding particular underscores
showstringspaces=false,         % underline spaces within strings
showtabs=false,                 % show tabs within strings adding particular underscores
frame=single,           % adds a frame around the code
tabsize=2,          % sets default tabsize to 2 spaces
captionpos=b,           % sets the caption-position to bottom
breaklines=true,        % sets automatic line breaking
%breakatwhitespace=false,    % sets if automatic breaks should only happen at whitespace
breakatwhitespace, % http://tex.stackexchange.com/questions/64750/avoid-line-breaks-after-lstinline
escapeinside={\%*}{*)},          % if you want to add a comment within your code
keywordstyle=\color{blue}\ttfamily,
stringstyle=\color{mymauve}\ttfamily,
commentstyle=\color{mygreen}\ttfamily
}

\lstset{inputencoding=utf8/latin1}
% Verhindert die Markierung!
%\lstset{
%literate={ö}{{\"o}}1
%          {ä}{{\"a}}1
%          {ü}{{\"u}}1
%          {ß}{{\ss}}1
%}

% currentname
\usepackage{nameref}
\makeatletter
\newcommand*{\currentname}{\@currentlabelname}
\makeatother

% remove line break after chapter
\usepackage{titlesec}
\titleformat{\chapter}[hang]
{\normalfont\huge\bfseries}{\chaptertitlename\ \thechapter }{0.5em}{}

% Fußzeile aktivieren
\pagestyle{fancy}
\fancyhf{}
\fancyfoot[R]{\thepage}
\fancyfoot[L]{\currentname}
\renewcommand{\headrulewidth}{0pt}
\renewcommand{\footrulewidth}{0.4pt}

% Redefine the plain page style
% plain is always used by the chapters
% http://tex.stackexchange.com/questions/117328/fancyhdr-does-not-apply-same-header-footer-on-chapter-and-non-chapter-pages
\fancypagestyle{plain}{%
 \fancyhf{}
 \fancyfoot[R]{\thepage}
 % Der Kapitelname ist auf der Seite des Kapitels nicht wichtig, daher weg lassen
% \fancyfoot[L]{\currentname}
 \renewcommand{\headrulewidth}{0pt}
 \renewcommand{\footrulewidth}{0.4pt}
}

% disable indentation
\setlength{\parindent}{0pt}
% disable ``chapter'' expression
\renewcommand{\chaptername}{}

% disable "chapter" expression completely
\addto\captionsngerman{\renewcommand{\chaptername}{}}

%\chapterstyle{section}

% http://tex.stackexchange.com/questions/103119/picture-on-title-page
%\titlehead{\centering\includegraphics[width=6cm]{logo.jpg}}

% \cite oder \cite
% http://citeseerx.ist.psu.edu für Zitate
% Alternativer Titel: Entwicklung einer C++-Schnittstelle mit austauschbarem Scheduler für die nebenläufige Ausführung von Tasks

\title{The Power of Fire Developer Manual}
\author{Tamino Dauth}

\begin{document}

%\bibliographystyle{alphadin}

% http://tex.stackexchange.com/questions/10130/use-the-values-of-title-author-and-date-on-a-custom-title-page
\makeatletter
\begin{titlepage}
\begin{center}
        {\Huge \@title}\\[2cm]
		 A Warcraft III: The Frozen Throne Modification\\
         \@author
        \\[0.5cm]
        \@date
\end{center}
\vfill

\end{titlepage}
\makeatother

%\maketitle


\newpage

%\cleardoublepage
%\phantomsection
%\addcontentsline{toc}{chapter}{Inhaltsverzeichnis}

%\addtocontents{toc}{\setcounter{tocdepth}{-1}}
%\chapter*{Inhaltsverzeichnis}
%\addtocontents{toc}{\setcounter{tocdepth}{1}}

\pagenumbering{Roman}
%\setcounter{page}{0}
\tableofcontents

% http://tex.stackexchange.com/questions/2995/removing-page-number-from-toc
\thispagestyle{empty}
\addtocontents{toc}{\protect\thispagestyle{empty}}

%\usetikzlibrary{patterns}


\newpage

\pagenumbering{arabic}

\chapter{Introduction}

The Power of Fire (German \enquote{Die Macht des Feuers}) is a modification of the realtime strategy game Warcraft III: The Frozen Throne. It alters the game to a roleplay game which can either be played in multiplayer or in a singleplayer campaign traveling between multiple maps.


Why should you create a map using The Power of Fire instead of creating your own map?
Here are a few good reasons:
\begin{itemize}
\item Finished Roleplaygame systems: You have a complete roleplaying game which is tested and should work.
\item Many great resources ready to use: Many great realtexture models are already implemented as Doodads and can be used. Besides many better looking ground textures can be used, too.
\item Existing Content: There's a huge amount of spells and reference content in the original modification.
\item Contribute to an Open Source project: Everything is Open Source. Your contribution may become part of the modification and might get improved. The project can grow with more contributors like you.
\end{itemize}

\newpage

\chapter{vJass}

vJass is a scripting language based on Warcraft III's scripting language JASS. It adds new features like object oriented programming to the scripting language and is implemented by several compilers which translate the vJass code into JASS code.

The first and probably most popular compiler is the JassHelper. It has also been used for The Power of Fire.

By now other approaches with a better syntax exist like Wurst which has not been there when the project was started. It is highly unlikely that the modification will change its core scripting language since too much code is based on it. Besides it has been tested with a specific version of the JassHelper, so there would be a risk of losing functionality or even missing features in the new scripting language.

The size of the modification helped to find the limits of vJass and to make usage of nearly all features. I even wrote many posts on Wc3C.net to improve the language and to report bugs.

\newpage

\chapter{MPQ}

Warcraft III uses the format MPQ for custom data archives. Therefore The Power of Fire provides a custom file called \enquote{TPoF.mpq} with all required resources such as models, textures, icons and sound files.

This file has to be loaded in the game and in the World Editor.

\newpage

\chapter{JassNewGenPack}

\newpage

\chapter{Creating a new Map}

\section{Globals}

Every map has to provide several globals which configure the map's systems:

\section{MapData}

Every map has to provide a struct called \lstinline[]{MapData} with several methods and static constants which are used by the Game backend of the modification to run all required systems.

\section{Required Rects}

\section{Required Camera Setups}

\newpage

\chapter{Spawn Points}

The Power of Fire allows you to respawn creeps and items on the map after being killd or picked up. This should improve the gameplay experience since after clearing a map returning to this map would be less interesting without any new creeps or items to collect.
Besides the player would get a disadvantage since he cannot level his character nor collect valuable stuff.

Usually the spawn points are created during the map initialization using existing units and items already placed on the map.

\newpage

\chapter{Translation}

To translate all maps as well as the campaign into different languages, one has to extract the war3map.wts files (before optimizing them out). After extracting the files, the entries have to be replaced by strings in another language. A copy of the unoptimized map must be created. Then the modified war3map.wts files have to be readded to the copies of the maps. If the maps are optimized afterwards (both, the one for the original language and the translated), they will differ and on online games won't be considered the same map only translated but the string entries will be optimized and the loading will become faster.

\vspace{\baselineskip}

The campaign file has to be copied and uses the translated maps. Besides the information etc. has to be translated.

\vspace{\baselineskip}

Besides the file for the user interface has to be replaced.

\newpage

\chapter{Icons}

The grimoire icons require an icon with every level from 0 to 6. There is an ability per level for the grimoire since changing the icon of an ability cannot be done dynamically. The script \emph{Scripts/dmdf-all-grimoire-icons} creates all those icons using ImageMagick. Since ImageMagick cannot handle BLP files. The icons have to be converted into PNG or TGA files.

\newpage

\chapter{Release}

\section{Optimizing}

Each map can be optimized using some standard routines. First of all the wc3lib can be used (wc3object) to drop all object data modifications which are not required anymore. This can happen when a hero ability is change to a unit ability but some hero only fields are still changed. This optimization reduces the number of strings which have to be translated or optimized later.

\vspace{\baselineskip}

Besides all object data fields which are for the World Editor only can be optimized. These are usually editor only suffixes. The number of modifications (size of the object data) and string entries will be reduced by this which should improve the loading speed of the map.

\newpage

\cleardoublepage
\phantomsection
\addcontentsline{toc}{chapter}{List of Figures}
\listoffigures

\newpage

%\cleardoublepage
%\phantomsection
%\addcontentsline{toc}{chapter}{List of Tables}
%\listoftables
%
%\newpage

\cleardoublepage
\phantomsection
\addcontentsline{toc}{chapter}{Listings}
\renewcommand\lstlistingname{Listings}
\renewcommand\lstlistlistingname{Listings}
\lstlistoflistings

\newpage

\setcounter{biburlnumpenalty}{9000}
\setcounter{biburlucpenalty}{9000}
\setcounter{biburllcpenalty}{9000}
\cleardoublepage
\phantomsection
\addcontentsline{toc}{chapter}{Bibliography}
\printbibliography[title=Bibliography]

\end{document}